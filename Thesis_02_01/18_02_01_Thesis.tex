\documentclass{TUD_NA_PhD_thesis}

\title{POD-based deflation method for reservoir simulation}

\usepackage{lipsum}
\usepackage[utf8x]{inputenc}
\usepackage{amsmath}
\usepackage{amsfonts}
\usepackage{mathrsfs}
%\usepackage{natbib}
\usepackage{graphicx} % figuras
%\usepackage[export]{adjustbox} % loads also graphicx
\usepackage{float}
\usepackage[font=footnotesize]{caption}
\usepackage{wrapfig}
\usepackage{authblk}
\usepackage{subfigure}
\usepackage{setspace}
\usepackage{amssymb}
\usepackage{latexsym}
\usepackage{footnote}


\begin{document}

\frontmatter

\maketitle

\tableofcontents

\chapter{Samenvatting}

\chapter{Summary}


\mainmatter

\chapter{Introduction}
\section{Reservoir simulation}
Petroleum reservoirs are layers of sedimentary rock, which vary in terms of grain size, mineral and clay contents. 
Reservoir simulation is a way to analyze and predict the fluid behavior inside a reservoir through the analysis of a model, which can be a geological or a mathematical model. \par 
The geological model describes the reservoir, i.e., the rock formation. For this description, a set of petrophysical properties are defined. The main properties are: the \emph{rock porosity}, $\phi$, defined as the fraction of void space inside the rock and the \emph{rock permeability}, $\mathbf{K}$, that determines the rock's ability to transmit a fluid through the reservoir. The rock permeability in general is a tensor where each entry $\mathbf{K}_{ij}$ represents the flow rate in one direction ($i$) caused by the pressure drop in the same ($i$) or in a perpendicular ($j$) direction. \par
For the mathematical modeling, we describe the flow through porous media making use of the principle of mass conservation and Darcy's law, corresponding to the momentum conservation.
The mass balance equation for a fluid phase $\alpha$ is given by:
\begin{equation}\label{eq:mc}
 \frac{\partial(\phi \rho_{\alpha}S_{\alpha})}{\partial t}+\nabla \cdot ( \rho_{\alpha} \mathbf{v}_{\alpha})=\rho_{\alpha} q_{\alpha},
\end{equation}
and the Darcy's law reads:
\begin{equation}\label{eq:D}
\mathbf{v}_{\alpha}=-\frac{\mathbf{K}_{\alpha}}{\mu_{\alpha}} (\nabla p_{\alpha}-\rho_{\alpha} g \nabla d),
\end{equation}
where, $\rho_{\alpha}$, $\mu_{\alpha}$ and $p_{\alpha}$ are the density, viscosity,  and pressure of the fluid;  $g$ is the gravity constant, $d$ is the depth of the reservoir and $q_{\alpha}$ are sources, usually, fluids injected into the reservoir. 
The saturation of a phase $(S_{\alpha}),$ is the fraction of void space filled with that phase in the medium, where a zero saturation indicates that the phase is not present.
Fluids inside a reservoir are usually filling completely the empty space, this property is expressed by the following relation:
\begin{equation}\label{eq:sr}
 \sum_{\alpha}S_{\alpha}=1.
\end{equation}\par
If our system consists in more than one fluid phase, the permeability of each fluid phase, $\alpha$, will be affected by the presence of the other phase. Therefore, the effective permeability $\mathbf{K}_\alpha$ has to be used instead of the absolute permeability $\mathbf{K}$.  The absolute and effective permeabilities are related via the saturation-dependent relative permeability:
$$\mathbf{K}_{\alpha}=k_{r\alpha}(S_{\alpha})\mathbf{K}.$$





\par
The fluid density $\rho_{\alpha}=\rho_{\alpha}(p)$ and the rock porosity $\phi=\phi(p)$ can be pressure dependent. The dependence is given by the \emph{rock compressibility}, $c_r$, 
\begin{equation}\label{eq:rc}
 c_r=\frac{1}{\phi}\frac{d\phi}{dp}=\frac{d(ln(\phi))}{dp},
\end{equation}
 for the porosity, and the \emph{fluid compressibility}, $c_f$,
\begin{equation}\label{eq:fc}
 c_f=\frac{1}{\rho_{\alpha}}\frac{d\rho_{\alpha}}{dp}=\frac{d(ln(\rho_{\alpha}))}{dp},
\end{equation}
for the density.


\subsection{Single-phase flow}
\subsection{Two-phase flow}

%\chapter{Thesis Overview}




\chapter{Numerical methods}
\chapter{Single-phase flow}
\chapter{Two-phase flow}

\backmatter
\setcounter{chapter}{0}

\chapter{Conclusions}


\appendix
\chapter{Appendix 1}\label{a1}
\section*{List of notation}
\begin{table}[!h]
\centering
\begin{tabular}{c l l }
\hline
Symbol & Quantity & Unit \\[0.5ex]
\hline
$\phi$ & Rock porosity&   \\
$\alpha$ & Fluid phase& \\
$\mathbf{K}$& Rock permeability&  $Darcy$ $(D)$ \\ 
$\mathbf{K}_{\alpha}$& Effective permeability  &  $Darcy$ $(D)$ \\
$\mathbf{k}_{r\alpha}$& Relative permeability  &  \\
$\mu_{\alpha}$&Fluid viscosity & $Pa \cdot s$   \\
$S_{\alpha}$  &Saturation &  \\
$\rho_{\alpha}$ &Fluid density &  $kg/m^3$ \\
 $\mathbf{v}_{\alpha}$ & Darcy's velocity& $ m/d$ \\
 $q_{\alpha}$ &Sources &   \\
  $c_r$& Rock compressibility&  $Pa^{-1}$ \\
$c_f$ &Fluid compressibility &  $Pa^{-1}$ \\
$g$  &Gravity &  $m/s^2$ \\
$d$ & Reservoir depth& \\




 $\lambda_{\alpha}$&Fluid mobilities& $D/ (Pa \cdot s)$   \\
${p}$  &Pressure &  $Pa$ \\
$p_{c}$  &Capillary pressure &  $Pa$ \\


\hline
\end{tabular}\label{table:symbols}
\caption{Notation}
\end{table}

\chapter{My 2 Appendix}
\chapter{My 3 Appendix}
\printbibliography

\end{document}
